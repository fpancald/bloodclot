
\section*{Introduction}
\label{sec:intro}


 It is becoming increasingly vital to combine
laboratory research with mathematical modeling and computer
simulation to get additional insight into biological processes
\cite{AssHer06,Arn06,CNX,HRAMMA,LGWNC,
LZV,MCNJY,PWSNCM,SJKA1,SJKA2,SCBM,TPMA,WJKA}. %To build a
%mathematical model that describes a biological process, the
%biological process is conceptualized by mathematical
%representations, e.g., differential equations, discrete models and
%stochastic equations.
In the process of model development, conceptual model is firstly
established by making some simplifying
assumptions. Then, it is implemented mathematically and analyzed using
theoretical approaches or numerical methods. Meanwhile,
experiments are used to estimate model parameters
and to calibrate
 and validate the model. However, due to errors in experimental
measurements, there always exists uncertainty in parameter
estimation. This uncertainty includes model simplifications,
computational errors coming from numerical schemes, and random local
rules in Monte carlo simulations. The situation could become even
worse if certain parameter values may be impossible to measure using
current experimental techniques, and to test a range of parameter
values. %Typically, data-fitting is used to estimate these parameters
%in this scenario.
Thus, identifying the sensitivity and uncertainty
of the model parameters becomes essential for using models to make
predictions and to explore biological hypotheses.


Recently, many methods have been developed for analyzing sensitivity of chemical/biological reaction networks. These include Morris method ~\cite{Morris91}, the sparse grid
probabilistic collocation method (SGPCM) \cite{DBJSH, LinAMTAWR,
LinAMTJSC}, pathwise sensitivity analysis methods \cite{PKV13} and many other methods.
However, many of these methods have difficulty in studying high-dimensional models.

In this paper, we introduce a new general framework for quantifying
uncertainty, analyzing the parameters, and determining steady states
of chemical/biological reaction network models in the form of systems of ordinary differential equations (ODEs) (see Fig. \ref{fig:Dia}),
and demonstrate this
approach by using it to analyze a specific example, Hockin-Mann's model \cite{HocJon02} of blood coagulation. 

Using this framework, we first
compute the steady state solution structures and characteristics of
a ODE model using a newly developed numerical algebraic
geometry method, which is capable of  finding all steady state solutions
 for arbitrary initial conditions. The steady state solutions are known to be dependent on
the initial conditions and reaction rates. We demonstrate that our new algorithm
is able to compute all steady states of models
 represented by a singular sparse polynomial system, and with a large number of
variables.


We then used two stochastic  sensitivity analysis tools, namely, Morris method ~\cite{Morris91} based on Monte
Carlo (MC) simulation and sparse grid collocation based on SGPCM
approach, to rank the importance of the reaction rate constants of the model.
We show that the results obtained by these two methods are consistent.
However, the SGPCM has higher order of convergence than
the  Morris method, which is a classical method on sensitivity analysis, and
depends on MC method. In addition, the SGPCM can also provide the effects
between reaction rates and initial concentrations of reactants respectively.

The SGPM is used in the end to quantify uncertainties of the model. We validate the result obtained by the SGPM with the one obtained by Monte Carlo to show that the SGPM can  handle the high-dimensional model.


We  examine the effectiveness of this computational framework by applying it to analyze a blood coagulation model,  Hockin-Mann's model \cite{HocJon02}.  This model simulates blood coagulation
reactions, one of the key processes controlling blood clot (or
thrombus) formation. And proper blood clotting (or hemostasis) is critical to
preventing hemorrhaging. Particularly, we explored the solution structures of the model,
ranked the  sensitivity and quantified the uncertainties of reaction rate constants of the model,  which
were determined by fitting computational results with empirical
data.

Most of the therapeutic drugs treating hemophilia would vary one or
two reaction rates or increase concentrations of certain coagulation factors to increase factor thrombin production. Using our sensitivity analysis results, we
studied the effects of several hemophilia drugs (shown in Table
\ref{tab:trial}) on the rate of thrombin production. Specifically, we
predicted the threshold value of concentration of factor VIII (FVIII) in reducing
the risk of bleeding disorder while keeping other blood coagulation factors at the normal levels of concentration. In this case, we predicted that it is sufficient to
 maintain the FVIII concentration to be 4.7\% of its normal level .



