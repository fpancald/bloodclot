\section*{Discussion}
\label{sec:conc} This paper presents a general mathematical and
computational framework which could also be used to identify effective drug therapy and
predict its performance with confidence interval (Figure
\ref{fig:Dia}).  First, an algorithm is developed to solve the
steady state system with rank-deficiency derived from a mathematical
model and establish a quantitative understanding of effects of
 reaction rates and initial conditions on the outputs of the model. This algorithm  is the first one we are aware
of that utilizes methods of numerical algebraic geometry to solve
rank deficient systems. It can be used to analyze other dynamic
systems with rank-deficiency. Moreover, our results show that the
steady states depend not only on initial conditions but also on
reaction rates. This algorithm enables us to find all solutions,
among which the time marching method would not be able to solve the
dynamical system with multiple equilibriums for given initial
condition.

Second, we have developed a new technique to quickly solve
the steady state system, as well as analyze the chemical reaction
model using sensitivity analysis and uncertainty quantification methods. 
Finally, effective drug therapy
is identified by comparing the importance ranking of the
corresponding reaction rates obtained through sensitivity analysis
study. We employ the numerical model to predict the performance of
the most effective drug therapy with confidence interval of
corresponding thrombin production using SGPCM. Further analysis will
be required to assess the effects of all parameters, such as
reaction rates, using compressive sensing and optimization under
uncertainty. Optimal control on the parameters for manipulating the
total thrombin will be derived in our future works.

%Hockin-Mann's model that describes the tissue
%factor mediated pathway of blood coagulation was used to demonstrate the effectiveness %of the framework. A new algorithm based
%on numerical algebraic geometry is proposed to compute the steady
%state system with rank-deficiency and is applied to the
%Hockin-Mann's model.

For Hockin-Mann's model,   we explored the
effects of $16$ reaction rates whose values are obtained by fitting
data and initial conditions. Both variance decomposition based
sensitivity analysis and Morris design method were performed to rank
the significance of the $16$ uncertain reaction rates with respect
to the total thrombin. Ranks of the importance of each reaction rate
with respect to total thrombin concentration and the interactions
between pair reactions are illustrated using a network graph (See
Figure \ref{fig:NSA}). Good agreement on the sensitivity ranking of
the reaction rates is observed using both methods. SGPCM is employed
to study the effect of the $16$ uncertain reaction rates on the
total thrombin concentration. The numerical results are consistent
with the results obtained from classic Monte Carlo simulations.
Additionally, the numerical results indicate that SGPCM can achieve
much faster convergence than classic Monte Carlo method with much
lower computational cost. (Figure 7)
\\
Computational studies support the following three main conclusions:
\begin{itemize}
\item  the steady states depend on the initial concentrations of
factors and chemical reaction rates;
\item  our methods compute all steady states, which couldn't be founded by time marching method;
\item  among $16$ fitted rate constants, the reaction rate $k_{14}$ plays an important role in generating the total
thrombin;
\item  the threshold concentration of factor VIII is controlled by
$k_{14}$ and factor X. This provides clinical guidance for the
examination of bleeding disorder.
\end{itemize}

All ``important" reaction rates, factors are listed in Box 2.

\begin{algorithm}[H]
\SetAlgoLined
 \SetAlgorithmName{Algorithm}{problem}{List of problems}
\SetKwInOut{Input}{Input}\SetKwInOut{Output}{Output} %\LinesNumbered
{\bf Box 2} Important reaction rates and factors
\begin{itemize}
\item {\bf $k_{14}$ and $k_{15}$}: are the most important reaction rates in
generating IXa=VIIIa and IXa=ATIII respectively, eventually play
important roles in generating the total thrombin;
\item {\bf factors VIII and IX}: deficiency causes two types of hemophilia
diseases;
\item {\bf $k_{14}$ and factor X}: control the threshold of factor
VIII, and reduce the risk of bleeding disorder.
\end{itemize}
\end{algorithm}


Although previous studies showed that experimental coagulation
initiation at different levels of TF effects the total thrombin
production. Our analysis investigates how altering both the initial
concentrations of pro-coagulant factors and values of reaction rate
constants can effect thrombin production, and shows that $k_{14}$
and $k_{15}$ play a major role in generating the total thrombin
since they directly control generation of the complex TF=VIIa=IX and
thereafter control the generation of factor IXa. On the other hand,
rates $k_{37}$ and $k_9$, which consume TF=VIIa complex and factor
X, prevent the total thrombin production. Moreover, $k_{4}$
activates, and $k_{37}$ inhibits the role of $k_{14}$. In fact,
larger values of $k_4$ yield more generation of TF=VIIa, which
speeds up the thrombin production, and then activates the role of
$k_{14}$. While $k_{37}$ inhibits the thrombin production by
consuming TF=VIIa to generate the complex TF=VIIa=Xa=TFPI, and then
gives negative feedback to $k_{14}$. Furthermore, blocking $k_{14}$
and $k_{15}$ results in significant sensitivity shifts. Figure
\ref{Fig:NSAD} shows that their affinities are predicted to be less
important when values of $k_{14}$ and $k_{15}$ are reduced to 10\%
of their normal values. Conversely, the sensitivity of $k_{23}$
related with factor VIIIa is found to increase.
 Our analysis also shows that the factor VIII plays an
important role in the examination of bleeding disorder for women who
are hemophilia carriers, and can vary within a wide range, namely,
20-80\% of the normal value. Analysis of the threshold concentration
of factor VIII can be used to predict if a carrier is at high risk
of bleeding disorder. The threshold level for this particular factor
in a patient would depend upon the activity rate and the
concentration of all coagulation factors. The current common
practice for hemophilia treatment is to replace the missing factor
with exogenous material. If it is a human derived factor VIII, this
carries the risk of spreading disease (as in the case of HIV and the
high number of hemophiliacs who developed AIDS). One could provide
recombinant proteins, but these are very expensive. Also recombinant
proteins can induce the patient to generate an immune response to
the foreign proteins and block their effectiveness. Thus,
identifying the threshold value of factor VIII enables the use of
small molecule inhibitors which are less likely to be immunogenic
and should be cheaper than production of recombinant proteins. The
sensitivity analysis could provide important clinical information to
direct therapy for bleeding and thrombotic disorders.

Furthermore, the sensitivity analysis can identify new targets for
hemorrhagic disorders. One approach to treat coagulation factor
deficiencies is to replace the missing factor with exogenous
material. Prophylactic use of exogenous material to prevent disease
is very expensive and exogenous material is often just provided
during the bleeding episodes but not for prophylactic purpose.
Furthermore, the exogenous material may induce an immune response
antagonizing the effect of replacement therapy. Sensitivity analysis
may help identify other elements of the coagulation network where
changes in activities may reduce the threshold concentration of the
missing factors. Thus small molecule inhibitors or activators of the
other elements (which are less likely to induce an immune response)
can reduce the effects of the factor deficiency. This potentially
may provide new therapeutic targets to treat coagulation
deficiencies, and it also may be possible to individualize the
treatment for a specific patients based upon their level of
coagulation components based on the guidance of the sensitivity
analysis.
