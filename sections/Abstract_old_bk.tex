{\bf Background:} The blood coagulation system is composed of a
complex network of chemical reactions. This network can be modeled
by a Hockin-Mann et al theoretical model.
\\{\bf Results:}
We have used a systematic approach to analyze how simulation fitted
reaction rate constants in the Hockin-Mann model of the extrinsic
(TF mediated) coagulation pathway affect rates of thrombin
production. We first present algorithms based on numerical algebraic
geometry to compute steady state solutions for different initial
conditions. By applying these algorithms to general rank-deficiency
systems, steady states which cannot be obtained by the time marching
method are determined. Stability of these steady state solutions is
studied using mathematical analysis and numerical simulations. We
then perform both variance decomposition based on the sensitivity
analysis and Morris design method to rank the significance of the
$16$ data-fitted reaction rates with respect to the total thrombin
production. The sensitivity ranking of the reaction rates show
agreement between both methods. Subsequently, the sparse grid
probabilistic collocation method (SGPCM) is employed to quantify how
uncertainties of these $16$ reaction rates influence total thrombin
production. Numerical results obtained by SGPCM are consistent with
the results obtained from the classic Monte Carlo simulations. These
methods can be applied to analysis of mathematical model with large
scale parameters. Additionally, the numerical results indicate that
SGPCM can achieve much faster convergence than classic Monte Carlo
method with much lower computational cost. Finally, we ranked the
importance of different reaction rate constants using sensitivity
analysis to identify critical threshold levels for factors VIII and
IX for hemophilia therapy.
\\{\bf Conclusions:} %Our simulations demonstrate that the
%equilibrium of the blood coagulation network depends upon not only
%the initial concentrations of factors in the blood, but also values
%of reaction rate constants.
We present a general framework for anlayzing reaction network
represented by system of ordinary differential equations (ODEs). In
these systems, there exists a large number of parameter sets.
Especially, in Hockin-Mann model, among these $16$ fitted reaction
rate constants, the reaction rate $k_{14}$, which is related to
factor IX, plays most important role in determining the rate of
thrombin generation. Moreover, the sensitivity analysis provides
threshold values for individual factors or provided reaction rate
constants in controlling thrombin production. Identifying the
threshold concentration of factor VIII provides clinical information
to direct therapy for hemophilia patients experiencing bleeding
disorder. In our simulation, we suggest that increasing the value of
$k_{14}$ and decreasing the concentration of factor X would provide
a feasible therapy for patients of bleeding disorder.
