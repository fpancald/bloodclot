
\section*{Introduction}
\label{sec:intro}


 It is becoming increasingly vital to combine
laboratory research with mathematical modeling and computer
simulation to get additional insight into biological processes
\cite{AssHer06,Arn06,CNX,HRAMMA,LGWNC,
LZV,MCNJY,PWSNCM,SJKA1,SJKA2,SCBM,TPMA,WJKA}. %To build a
%mathematical model that describes a biological process, the
%biological process is conceptualized by mathematical
%representations, e.g., differential equations, discrete models and
%stochastic equations.
First, conceptual model is established by making some simplifying
assumptions. Then, it is implemented mathematical and analyzed using
theatrical approaches or numerical methods. During the course of
model development, experiments are used to estimate model parameters
and to calibrate
 and validate the model. However, due to errors in experimental
measurements, there always exists uncertainty in parameter
estimation. This uncertainty includes model simplifications,
computational errors coming from numerical schemes, and random local
rules in Monte carlo simulations. The situation could become even
worse if certain parameter values may be impossible to measure using
current experimental techniques, and to test a range of parameter
values. %Typically, data-fitting is used to estimate these parameters
%in this scenario.
Thus, identifying the sensitivity and uncertainty
of the model parameters becomes essential for using models to make
predictions and to explore biological hypotheses.

The Hockin-Mann's model \cite{HocJon02} simulate blood coagulation
reactions, one of the key processes controlling blood clot (or
thrombus) formation. Blood clotting (or hemostasis) is critical to
preventing hemorrhaging. Prior to the Hockin-Mann's model
\cite{HocJon02}, Mann's group proposed in the form of system of
ordinary differential equations \cite{JonMan94a,JonMan94b} to
describe the TF-initiated pathway reactions based on \emph{in vitro}
experiments. Despite wide applications of Hockin-Mann's model
\cite{JonMan94a}, structures, characteristics of steady state
solutions of the model, sensitivity of the model, and uncertainty of
parameter values have
yet to be discussed. %In fact, most coagulation models are rarely
%analyzed.
 So far,  only Luan \emph{et al.} \cite{LZV}  applied sensitivity analysis to a network of
coagulation reactions to identify fragile sites. Using parameter
values from the literature, reactions were identified where small
changes in parameter values would have dramatic effects on thrombin
generation and platelet activation. This analysis \cite{LZV}
 identified reactions involving an interaction of factors IX and VIII
as the most sensitive reactions to small fluctuations of relevant
parameters.

In this paper, we introduce a new general framework for quantifying
uncertainty, analyzing the parameters, and determining steady states
of biological models (see Fig. \ref{fig:Dia}), then demonstrate this
approach by using a specific example (Hockin-Mann's model). We first
compute the steady state solution structures and characteristics of
Hockin-Mann's model \cite{HocJon02} using numerical algebraic
geometry method. We present a new numerical algebraic geometry
algorithm to find all steady state solutions
 for arbitrary initial conditions. The steady state system of the model
\cite{HocJon02} is a singular sparse polynomial system. To deal with
the singularity, we propose a new algorithm for a large number of
variables. The steady state solutions are known to be dependent on
the initial conditions and reaction rates. Exploring these solution
structures and characteristics provides insights into understanding
the inventory of the proteins and the associated biophysical and
enzymatic processes involved in blood clotting. Moreover, analysis
and control of threshold of factor VIII shows that the therapy of
haemostasis controlling $k_{14}$ and factor X can reduce the risk of
bleeding disorder.

There are $16$ reaction rates of the model \cite{HocJon02} which
were determined by fitting computational results with empirical
data. Evaluation of uncertainties of these reaction rates has not
been conducted yet. Moreover, sensitivity of these reaction rates
has not been studied to the best of our knowledge. However, these
parameters' study would provide a feasible therapy (involving the
most important reaction rates), and simplify the mathematical model.
We next apply sensitivity analysis and uncertainty quantification to
rank the importance of the reaction rates, and to quantify their
interactions. Variance of these reaction rates on thrombin
production and covariance of any pair of these reaction rates on
thrombin production are decoupled and plotted in Figure
\ref{fig:NSA1}. Among these fitted rate constants, we identified
that the reaction rates ($k_{14}$ and $k_{15}$, associated with
factor IX, shown in Table \ref{taba}) associated with factor IX play
the most important roles in regulating the rate of thrombin
production. Specifically at 20\% variation in $k_{14}$ and $k_{15}$
can increase thrombin production by 200\%.

Most of the therapeutic drugs treating hemophilia would vary one or
two reaction rates. Using our sensitivity analysis results, we
studied the effects of several hemophilia drugs (shown in Table
\ref{tab:trial}) on the rate of thrombin production. Moreover,
analysis of the threshold value of factor VIII (FVIII) could provide
useful information to treat hemophilia, and reduce the risk of
bleeding disorder: using normal levels of all coagulation network
components (except for FVIII) the threshold FVIII concentration is
4.7\% of normal levels. We determined the that the reaction rate
$k_{14}$ and the factor X (FX) concentration have dramatic effects
on thrombin generation. By increasing the value of $k_{14}$ and the
concentration of FX, the threshold concentration of FVIII is reduced
to 3.3 of normal. This technique could be valuable in identifying
hemophilia carriers who are at risk of bleeding. FVIII levels in
hemophilia carriers can very widely from 10 - 90\% of normal values.
By determining the level of other factors and rate constants, one
can identify whether an individual carrier has FVIII levels near
their critical threshold and thus may be at risk of uncontrolled
bleeding.
