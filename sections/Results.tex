\section*{Results}
\label{sec:results}

\subsection*{Mathematical and computational framework for analyzing biological reaction networks}
We present a general approach to compute the steady states and
analyze importance of parameters, and provide diagram of different
stages in the process in Figure \ref{fig:Dia} and Box 1.
We demonstrate this general approach by analyzing Hockin-Mann's model.


\begin{algorithm}[H]
\SetAlgoLined
 \SetAlgorithmName{Algorithm}{problem}{List of problems}
\SetKwInOut{Input}{Input}\SetKwInOut{Output}{Output} %\LinesNumbered
{\bf Box 1} Computing
steady states and analyzing the effects of parameters for a general reaction network model.
\begin{itemize}
  \item[1] numerical algebraic geometry
(NAG) introduces a novel paradigm of computing steady states.
Especially, we present a new algorithm for the rank-deficient
polynomial system which is derived from the reaction network model;
\item[2] uncertainty quantification (UQ) and sensitivity analysis (SA)
estimate and identify parameters in the mathematical model. In
particular, we employ SGPCM, Morris method to analyze the effects of
parameters;
\item[3] Adjusting the most sensitive parameters and comparing with the existence
therapy provide  threshold values of sensitive parameters for a given
disease.
\end{itemize}
\end{algorithm}
% Using simulations, we are able to validate and
%calibrate the mathematical model, and make predictions for
%biological applications. We use this approach to solve Hockin-Mann's
%model, and then provide an example to demonstrate general approach,
%and show outcome and benefits in specific biomedical application.
\subsection*{Analysis of steady state solutions}
\label{sec:res}

Using \textbf{Algorithm 2} described in \textbf{Methods} to solve the steady state system of the
blood coagulation model, we found $88$ equilibrium (steady state
solutions) for Hockin-Mann's model based on initial conditions from
\cite{HocJon02}, and that all equilibriums are determined by linear
and quadratic equations shown in Table \ref{steady}. In particular,
Thrombin IIa ( $x_{13}$) and meizothrombin mIIa ($x_{16}$) are free
variables, and linearly depend on the initial condition (linear
equations), reaction rates, and values of other species such as
TF=VIIa=X ($x_{17}$), TF=VIIa=Xa ($x_{19}$) as well as Xa=TFPI
($x_{20}$) (quadratic equations). Moreover, the constants shown in
linear equations, such as $3.3383\times10^{-9}$ and
$2.33\times10^{-10}$, are determined by initial conditions.
Quadratic equations account for effects of reaction rates $k_i$.

Next, linear stability analysis is used to explore the steady state
solution and study the matrix eigenvalue problems of the linearized
system. Moreover, the nonlinear stability analysis is applied to a
simplified steady state solution.

\subsubsection*{Linear stability}
We consider the local stability of steady state solutions. Assuming
that $x_i=x_i^s+\epsilon x_i^1+O(\epsilon^2)$ for $ i =
1,\cdots,34$, where $x_i^s$ is the $i$-th component of steady state
solutions we obtained using the method described in section {\bf
Methods}, and $x_i^1$ is the first order term by Taylor expansion at
the steady states. By substituting $x_i$ into Eq. (\ref{1}) in
Appendix B, and dropping $O(\epsilon^2)$ and higher order terms, the
linearized system is obtained and shown in Appendix D. The
linearized system can be written as
\[\frac{d\mathbf{x}^1}{dt}~=~A(\mathbf{x}^s)\mathbf{x}^1,\]
which implies that $\mathbf{x}^1=e^{A(\mathbf{x}^s)t}$. If all the
eigenvalues of $A(\mathbf{x}^s)$ are non-positive, the steady state
solution $\mathbf{x}^s$ is stable. Since the matrix
$A(\mathbf{x}^s)$ linearly depends on $\mathbf{x}^s$, we obtain the
following formula
\[A(\mathbf{x}^s)~=~\sum_{i=1}^{34} A_i x^s_i,\]
where $A_i$ is a real matrix and independent on $x^s_i$.

All of the maximum eigenvalues of $A_i$ are negative and are shown in
Table~\ref{eig}. Therefore all eigenvalues of $A(\mathbf{x}^s)$ are
also negative for any $\mathbf{x}^s$ with positive components. Thus,
 the linearized system corresponding to any steady state solution is
stable. Then, we switch our attention to check the nonlinear
stability of solutions shown in Table \ref{steady}.

\subsubsection*{Nonlinear stability}
Among these $88$ steady states, we analyze one steady state, which
is ``close" to results obtained by solving Hockin-Mann's model
\cite{HocJon02} using the time marching method. And rest of steady states can be analyzed in a similar way. Specifically, this
steady state solution is obtained by the following manner: combining
with the time marching result from \cite{HocJon02}, the free
variables in Table \ref{steady} are set to zero except the thrombin
($x_{13}$), meizothrombin ($x_{16}$) and TF=VIIa=X ($x_{17}$). Thus,
a simplified steady state solution is obtained in Table
\ref{steady2}.
By solving the polynomial system with respect to variables
$x_{13},x_{16}$ and $x_{17}$ with Bertini \cite{Bertini}, we get one
real positive solution as follows
\[
x_{13}=2.99990\times10^{-8},x_{16}=2.32829\times10^{-14},
x_{17}=1.03445\times10^{-12}.\] Then $x_{19}$ and $x_{20}$ can be
obtained by substituting $x_{16}$ and $x_{17}$ into the linear
equations in Table \ref{steady2}. This solution shows that total
thrombin concentration is related with the complex TF=VIIa=X
($x_{17}$). In particular, equation
$k_{18}x_{17}+k_{25}x_{17}-k_{19}x_{16}x_{13}=0$ reveals the fact
that TF=VIIa=X links with thrombin and meizothrombin through
IXa=VIIIa in No. 11 and 15 chemical equations. Moreover, TF=VIIa=X
also affects the amount of TF=VIIa=Xa ($x_{19}$) and Xa=TFPI
($x_{20}$) together with meizothrombin (linear equations).  We note that the difference between the solution computed by the
time marching method and the steady state solution that we choose to
analyze is smaller than  $10^{-7}$ in $L_1$ norm. Relative
differences of the nonzero components are shown in  Table \ref{RE}.

Nonlinear stability analysis is accomplished by the following steps:
applying random perturbations (a sequence $\epsilon_n$ in Table
~\ref{sta}) to the steady state solution gives initial conditions
for the initial value problem shown in Appendix B; Runge-Kutta
method is used to solve this initial value problem numerically. The
results, summarized in Table~\ref{sta}, demonstrate that the steady
state solution is non-asymptotically stable. In this case, the
system moves away in a small region from the steady state solution
after introducing small disturbances. Therefore, this chemical
reaction network depends sensitively on the initial condition and
the reaction rates.

\subsubsection*{Steady states for different initial
conditions} \label{sec:sa}

Since initial conditions determine linear equations by generating an
augmented system and play an important roll in computing steady
states, we consider different initial conditions from \cite{KF01},
which respond in a threshold manner to changes in the availability
of particular surface binding sites for enzymes and
zymogens. %An increase in TF binding sites, as would occur with
%vascular injury, changes the system's production of thrombin
%dramatically.
Using the {\bf
Algorithm 2}, %in section \ref{sec:steady}
 we obtained
steady state solutions and show them in Table \ref{steady21}. The
free variables in Table \ref{steady21} are set to zero, and then a
specified steady state solution is obtained in Table \ref{steady22}.
By solving the polynomial system in Table \ref{steady22}, we get two
positive real solutions as follows: \bes &\
&x_{13}=4.99656\times10^{-7},x_{16}=4.12563\times10^{-13},
x_{17}=3.43569\times10^{-10};\label{solution1}\\ &\
&x_{13}=4.46036\times10^{-7},x_{16}=7.25906\times10^{-11},
x_{17}=5.39634\times10^{-8}.\label{solution2}
 \ees
 In these two steady state solutions, (\ref{solution1}) can be
 obtained by a time marching method based on the given initial conditions, while (\ref{solution2}) can not
 be obtained by the time marching approach.
 However, (\ref{solution2}) is also stable since it can come back to itself by
giving small perturbation to (\ref{solution2}) by time-marching. In
many biological systems, there exists multiple equilibriums even for
a given initial condition. Moreover, a small perturbation to an
equilibrium can approach the equilibrium by time-marching indicates
 that the equilibrium is stable. Therefore, our method is
 competitive to find multiple steady state solutions.

Comparing Tables \ref{steady2} and \ref{steady22}, polynomial
equations have different coefficients. Therefore, these steady state
solutions show a very sharp transition in predicting thrombin
production as the initial conditions vary. Changes of these initial
conditions induce an increase of thrombin concentration ($x_{13}$)
by $10$ times. Moreover, meizothrombin concentration ($x_{16}$) is
also increased by $10$ or even $1000$ times. Our results are
consistent with that of \cite{KF01} and \cite{LoDen05}. In fact,
initial TF density is known to change the total thrombin
significantly when a vessel is injured. A threshold dependence on TF
concentration is found in the experiments by Veer and Mann
\cite{VM}. Additionally, numerical study for low TF density was
carried out in \cite{LoDen05} by using Kinetic Monte Carlo
simulation. However, the steady state solutions depends not only on
initial condition, but also on reaction rates. It is obvious that
the quadratic equation is related with rate constants $k_i$.
Understanding how variation of reaction rates affects steady or
dynamic solutions is difficult to discern using linear or nonlinear
stability analysis. We therefore perform sensitivity analysis and
uncertainty
quantification to the system and show their effects. %in Section
%\ref{sec:uq}.

\subsection*{Sensitivity analysis and uncertainty quantification of
reaction rates and initial conditions and their impacts on steady
state solutions} \label{sec:uq}

In Hockin-Mann's model, there are $16$ reaction rates (shown in
Table \ref{reactrate}) that are derived through fitting
computational results to empirical data.  In order to investigate
the uncertainties associated with these $16$ reaction rates and
depict their impacts on model predictions, the sparse grid
probabilistic collocation method (SGPCM) \cite{DBJSH, LinAMTAWR,
LinAMTJSC} is utilized to generate the collocation points on the
feasible parameter space of these $16$ reaction rates, which are
assumed to be random inputs with multivariate uniform distribution.
\subsubsection*{Sensitivity analysis of 16 data-fitted  reaction rates}
\label{sa}
% In general, it is computationally intensive to perform stochastic
%simulations with all inputs treated as random variables. Therefore,
%it is crucial to examine the output due to random inputs, and to
%rank inputs accordingly.
In this subsection, we apply two different stochastic sensitivity
analysis methods, Morris method ~\cite{Morris91} based on Monte
Carlo (MC) simulation and sparse grid collocation based on SGPCM
approach, to rank the importance of the reaction rates on the total
thrombin and study how the variation (uncertainty) in the total
thrombin can be attributed to variations in the reaction rates as
the inputs, i.e., systematically changing reaction rates in the
model to determine the effects of such changes on the total thrombin
concentration.

In applied statistics, the Morris method~\cite{Morris91} for global
sensitivity analysis is very efficient compared to other methods. We
used the ``Morris one at a time" (MOAT) module based on MC
simulation performed by PSUADE \cite{PSUADE}, which is a software
toolkit to facilitate the uncertainty analysis and design
exploration. Figure \ref{Fig:mmean} and Figure \ref{Fig:std} show
that the reaction rates $k_{14},k_{15}$ and $k_{11}$ have more
significant impact on the total thrombin synthesis compared with
other fitted reaction rates. This result is consistent with the
current understanding of coagulation cascade.
 $k_{14}$ is the rate to form complex TF=VIIa=IX
using TF=VIIa and IX. TF=VIIa=IX is subsequently disassociated to
TF=VIIa and IXa with rate $k_{15}$. Moreover, $k_{11}$ can control
the production of TF=VIIa and Xa from TF=VIIa=Xa. The formation of
the extrinsic factor Xa plays a key role in the total thrombin
production, since prothrombinase  catalyzes the conversion of
zymogen prothrombin to thrombin.

Next, we used sparse grid probabilistic collocation
methods~\cite{DBJSH, LinAMTAWR, LinAMTJSC}, which can also result in
sensitivity information, to rank sensitivities of these reaction
rates and compare the results with that obtained by the Morris
method~\cite{Morris91}. Total variance of reaction rates on
catalyzing thrombin was decomposed using analysis of variance
(ANOVA) technique as
\[v_{total}~=~\sum v_i ~+~\sum v_{ij}~+~\sum v_{ijk}~+~\cdots,\]
where $v_i$ is the variance of reaction rate $i$, $v_{ij}$ is the
covariance of rates $i$ and $j$, and $v_{ijk}$ is the covariance of
rates $i$, $j$ and $k$, etc.

This variance decomposition indicates the amount of information each
reaction rate ($v_i$) contributes to the total thrombin production
and determines how much of the variance of each reaction rate can be
explained by the other reaction rates ($v_{ij}$). We use a network
graph to present ranks of the importance of each reaction rate with
respect to the total thrombin and also illustrate the interactions
between 16 data-fitted reaction rates. See
 Figure \ref{fig:NSA1} in which the radius of disks
corresponding to reaction rate $i$ is proportional to
$v_i/v_{total}$, representing the normalized variance associated
with reaction rate $i$. The lines between any pair of reaction rates
$i$ and $j$ is proportional to $v_{ij}/v_{total}$, representing the
correlation between that pair. The thicker the line is, the larger
the correlation is. Figure \ref{fig:NSA1} also helps us to avoid the
mistake of ignoring reactions which have small variances associated
with themselves, but large correlation with other reactions. It is
obvious that $k_{14}$ and $k_{15}$ play  major roles in generating
thrombin. We note that only two-way interactions are shown in Figure
\ref{fig:NSA1}, but higher order interactions can also be displayed
using the same approach. We also note that, when the output is
switched, the sensitivity is also switched. We applied variance
decomposition and plotted the sensitivity analysis in Figure
\ref{fig:NSA} when output changed to IXa=VIIIa and IXa=ATIII. Figure
\ref{fig:NSA} shows that $k_4$ and $k_{14}$ play important roles in
generating IXa=VIIIa ($x_{33}$) and IXa=ATIII ($x_{30}$)
respectively.

%Morris design~\cite{Morris91} based on MC simulations is also
%performed on sensitivity analysis of 16 data-fitted reaction rates.
%The standard deviations in Figure \ref{Fig:mmean} and Figure
%\ref{Fig:std} show that $k_{14}$ is the most important reaction rate
%in generating the total thrombin,

We conclude that the results obtained by the Morris design based on
the MC simulations are consistent with the results based on SGPCM
(Figure \ref{fig:NSA1}). Both of them show that the variances of
$k_{14}$, $k_{15}$ and $k_{11}$ are larger, and that they are
dominant reaction rates in generating the total thrombin.

\subsubsection*{Sensitivity analysis of initial conditions}
\label{sec:sens_init} From pervious study, we showed that initial
concentrations of factors also contribute to steady states of the
blood coagulation model. %Although effect of tissue factor on total
%thrombin synthesize is studied in\cite{LoDen05}, effects of the
%other factors are rarely studied. The reaction rates are used the
%data fitting value in \cite{HocJon02}.
Here we performed sensitivity analysis of $10$ non-zero initial
concentrations of zymogen factors using SGPCM. These non-zero
initial concentrations of zymogen factors are assumed to vary in
80\%-120\% of their normal values, and follow a uniformly
distribution. The effects of initial conditions on the total
thrombin synthesis is shown in Figure \ref{Fig:nsaic}. Factor X
($x_{15}$) plays an important role in generating total thrombin and
its cofactor Xa interacts with TFPI ($x_{18}$), which inhibits the
thrombin production. Blood coagulation simulation is dependent on
various model components such as reaction rates, and initial
concentrations. Sensitivity analysis of these reaction components
can give a guidance to experimental variability in increasing and
decreasing thrombin production.

%Moreover, MC or SGPCM allows prediction of individual clotting
%scenarios as well as the standard deviation. The time marching
%method produces a single result, representing experimental averages.


\subsubsection*{Validation of uncertainty quantification method}
The SGPCM approach can achieve much faster convergence than a
classic MC method and it handles a larger number of uncertain
parameters with relatively lower computational cost, when comparing
sparse grids with full tensor products \cite{LinAMTAWR}. Here we
utilized the MC simulations to verify accuracy of the SGPCM
approach, which we employed to do uncertainty quantification. In
Figure \ref{Fig:comp}, we plotted the difference of the mean and
variance of the total thrombin obtained by SGPCM ($545$ points) and
MC ($1000$ points). The results (mean and variance of the total
thrombin) obtained by the SGPCM approach is consistent with that of
MC simulations. But the SGPCM is more efficient, and achieves faster
convergence than MC. We used three different levels of sparse grid
points ($184$ points, $1228$ points and $6430$ points) associated
with $16$ data-fitted reaction rates and treated the mean and
variance of $6430$ points as the ``real solution". The absolute
errors of the mean and variance for $34$ variables are shown in
Table \ref{conv} (Appendix E) which verifies the convergence of the
SGPCM approach.



\subsubsection*{Propagation of uncertainty of reaction rates}
\label{sec:uq}

Propagation of uncertainty of reaction rates in model solutions is
studied in this section by using initial conditions described in
\cite{HocJon02}.  The $16$ data-fitted reaction rates are input
variables and steady state concentrations of different factors are
output variables. Runge-Kutta method is used to compute the time
marching system on the sampling space of input variables.
Figure~\ref{Fig:steady} shows the mean and error bars of steady
state solutions whose magnitudes are greater than $10^{-10}$ $\mu$m.
Error bars show how the data are spread and encompass the lowest and
highest values by using formula $\hbox{\emph{mean}} ~\pm~
\hbox{\emph{standard deviation (SD)}}$, where \emph{SD} is the
average or typical difference between the data points and their
mean. It is obvious that the steady state solutions we found %in
%section \ref{sec:res}
are lying inside the intervals of Figure~\ref{Fig:steady}. The
results of the mean and error bar for time marching simulations of
undiluted whole blood triggered with 25 pM TF has been shown in
Figure \ref{Fig:mean}. Since it shows that thrombin production and
consumption reach equilibrium by $700$ seconds in Hockin-Mann's
model, we studied the simulation over $700$ seconds in time. Total
thrombin is defined as thrombin (IIa) and meizothrombin (mIIa) with
meizothrombin multiplied by a factor $1.2$ because of
meizothrombin's $20$\% greater activity toward substrates
\cite{HocJon02}. Figure \ref{Fig:mean} plots the mean and error bar
of the total thrombin with total $545$ sparse grid points for $16$
data-fitted reaction rates.  This prediction confidence interval of
total thrombin production is provided with the model as the
quantification of the uncertainty.


\subsection*{Prediction of hemophilia therapy targeting factors VIII \& IX}
The sensitivity analysis provides threshold values for individual
factors or reaction rate constants in controlling thrombin
production. One application of our study is to examine women who are
carriers of hemophilia diseases. Hemophilia is a group of hereditary genetic
disorders that impair the body's ability to control blood clotting
or coagulation, which is used to stop bleeding when a blood vessel
is broken. There are two types of hemophilia diseases: one is factor
VIII deficiency; and the other one is factor IX deficiency. These
carriers contain a normal and defective gene for factor VIII or
factor IX, and can present with a wide range of factor VIII or
factor IX concentration levels in blood; as wide as 0-20\% of
 concentrations in the blood of healthy individuals respectively. By knowing the threshold concentration
of factor VIII or factor IX given the concentration levels of other
factors, one can predict the level of thrombin production in case of
bleeding, and subsequently determine if a carrier is at risk of
bleeding disorder using our approach.

In this section, we show how factors VIII and IX quantitatively
control the total thrombin concentration to identify individuals at
risk of bleeding disorder for hemophiliac patients. We assume that
the amounts of factors VIII and IX vary in ranges of 0-20\% of their
normal values. We also assume that the threshold of the total
thrombin is 10\% of its normal value. Below this level, people would
have bleeding disorder. Other factors are assumed to be at their
normal levels of concentration. We use uniformly distributed
sampling points on factors VIII and IX, each factor having 100
points. The total thrombin is computed at each pair of sample points
of factors VIII and IX by solving the time dependent system (see
Appendix B). The total thrombin is compared to the assumed threshold
value of total thrombin. Figure \ref{Fig:F89} shows that the
threshold concentration of factor VIII is 4.7 \% of its normal
value, and that the threshold is increased when the concentration of
factor IX increases. The histogram in Figure \ref{Fig:F89hist} shows
the probability of the total thrombin above the threshold is 75.4\%
when factor VIII and factor IX vary. In this case, the risk of
bleeding disorder is high (24.6\%). The threshold of factor VIII is
4.7\% of the normal value.



Furthermore, this analysis also identifies other factors that might
lower the threshold concentration of factor VIII so that above
threshold level of thrombin is produced to prevent from bleeding
disorder. Based on the sensitivity analysis, we noticed that the
threshold concentration of factor VIII would be lowered by
increasing concentrations of the pro-coagulant factors that lead to
produce total thrombin. Figures \ref{fig:NSA1} and \ref{Fig:nsaic}
show that the total thrombin concentration is activated by raising
the value of reaction rate $k_{14}$ and the initial concentration of
factor X ($x_{15}$). Figure \ref{Fig:RV} shows that the total
thrombin production is increased from $0.145$ to $0.17$ as $k_{14}$
increases by 10\% of the normal value. Increasing the value of
reaction rate $k_{14}$ is to activate factor IX to its active form
IXa. This study is consistent with the study of the local measures
\cite{HM}, which are used with a pre-operative dose of raising the
concentration level of factor IXa. In \cite{HM}, factor IXa is
increased 10\% by using local measures of tranexamic acid solution,
and excellent haemostasis is achieved for severe hemophiliac
patients. In order to lower the threshold value of factor VIII in
Figure \ref{Fig:F89}, we let the rate of $k_{14}$ and the initial
concentration of factor X ($x_{15}$) vary in the range of 100-200\%
of their normal values. We again take 100 uniformly distributed
sampling points in each of their parameter value spaces. The total
thrombin is computed on the factor VIII and IX sampling space by
time marching method. Figure \ref{Fig:F89m} shows that the threshold
of factor VIII is decreased to 3.3\% by raising the value of
reaction rate $k_{14}$ and the initial concentration of factor X.
Figure \ref{Fig:F89mhist} shows the probability of bleeding disorder
is also decreased to 17.7\%. Thus it is possible by altering the
activity the level of reaction rate $k_{14}$, and the initial
concentration of factor X, one could reduce the level required of
factor VIII to maintain normal hemostasis. Moreover, Factor IX is an
important drug of haemostasis (see Table \ref{tab:trial} for a
sampling of current trials involving factor IX). Most trials involve
the important factor IX. The lack of factor IX can produce bleeding
episodes that cause damage of the bone, muscles, joints, and
tissues. Control mechanisms involving factor IXa are common
strategies in current anticoagulation conical therapies and clinical
trials \cite{LZV}.
