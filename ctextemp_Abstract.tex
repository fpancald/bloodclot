\def\CTeXPreproc{Created by ctex v0.2.11, don't edit!}{\bf Background:} The blood coagulation system is composed of a
complex network of chemical reactions. This network can be modeled
by the Hockin-Mann et al theoretical model.
\\{\bf Results:}

We have developed a systematic approach to analyzing high-dimensional stoichiometric
biological-reaction network models represented by systems of ordinary differential equations (ODEs) with parameters obtained by data-fitting.
In this approach, A new algorithm based on numerical algebraic
geometry  is introduced to compute steady state solutions to rank-deficient
systems with different initial
conditions.  In general, steady states to rank-deficient systems cannot be obtained by the time marching method.
Then the variance decomposition based on the sensitivity
analysis and Morris design method is used to rank the significance of the reaction rate constants obtained by data-fitting on affecting outputs of the models.
Subsequently, the sparse grid
probabilistic collocation method (SGPCM) is employed to quantify how
uncertainties of these reaction rates influence the model outputs.
\\{\bf Conclusions:} %Our simulations demonstrate that the
%equilibrium of the blood coagulation network depends upon not only
%the initial concentrations of factors in the blood, but also values
%of reaction rate constants.
We present a general framework for analyzing high-dimensional
biological reaction network models represented by system of ODEs with a large number of parameter sets.
We also show that that SGPCM can achieve much faster convergence than classic Monte Carlo
method with much lower computational cost.

Specifically, we systematically analyzed the Hockin-Mann model of the extrinsic
(TF mediated) coagulation pathway model using this framework.
We ranked the significance of the $16$ data-fitted reaction rates of the model on affecting the total thrombin production and found that the reaction rate $k_{14}$, which is related to
factor IX, plays the  most important role in determining the thrombin generation.
Finally, we ranked the
importance of all reaction rate constants using sensitivity
analysis to identify critical threshold levels for factors VIII and
IX to  suggest how to alter either concentrations of these factors or
 reaction rates associated with them for hemophilia therapies targeting these
 factors. Using simulations, we suggest that increasing the value of
$k_{14}$ or decreasing the concentration of factor X would be feasible for hemophilia patients with bleeding disorder.
