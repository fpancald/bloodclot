\def\CTeXPreproc{Created by ctex v0.2.11, don't edit!}\section*{Biological Background of Hockin-Mann's model}
\label{sec:back}


During the initial step of clot formation, resting platelets in
blood are activated following adhesion to matrix proteins exposed
from the damaged vessel wall. At the same time, blood borne factor
VII binds with vessel wall tissue factor (TF) to form VIIa-TF
complex (VIIa stands for activated factor VII) to initiate a complex
set of blood coagulation reactions \cite{Nem92}. Coagulation
reactions lead to the formation of thrombin from prothrombin.
Thrombin is efficient in activating platelets and also converts
fibrinogen in blood to fibrin which self-polymerizes, forming a
fibrin network that is a major
structural component of the clot \cite{HWL,LSK}. %(See Section
%\ref{sec:back} for details of the biological background).

Platelets lead to pro-thrombotic changes upon activation. These
include the release of their contents of alpha and dense granules
that contain coagulation factors and platelet activators,
modification of the platelet surface to promote surface dependent
coagulation reactions, and the activation of the integrin receptor
gpIIaIIIb which can bind fibrin(ogen), von Willebrand Factor (vWF)
and vitronectin, enabling these molecules to bind adjacent platelets
and increase platelet to platelet adhesivity.


In parallel with platelet activation, VIIa-TF complex formed by
factor VII and TF binding catalyzes the conversion of factor X to Xa
directly or via a IXa-VIIIa intermediary step. Activated factor Xa
combines with its protein cofactor, factor Va, on the surfaces of
platelets to form prothrombinase (Va-Xa complex) which, in turn
converts prothrombin (II) to thrombin (IIa) through an intermediate
meizothrombin (mIIa). Transient mIIa is rapidly converted to IIa in
the presence of factor Va and pro-coagulant lipids. Factor V is
activated to its active form Va by both factor Xa and IIa. Thrombin
activates platelets, converts fibrinogen to fibrin and activates
factors V, VIII, and XI which provide a positive feedback loop
propagating the production of thrombin. Thrombin is also a potent
activator of resting platelets and thus mediates recruitment of
resting platelets flowing nearby in the blood. The activated
platelets provide a procoagulant surface that promotes coagulation
enzyme activity. The formation and activity of VIIIa-IXa and Va-Xa
complexes are dependent on the availability of phospholipid binding
sites, which are only expressed on the surfaces of activated
platelets \cite{AssHer06,BPM,HocJon02,KhaSem89,KF01,LSK}.


The coagulation process is also regulated by inhibitors of
coagulation reactions. TFPI binds
 VIIa-TF with factor Xa resulting in inhibition of the
coagulation initiation reaction. ATIII inactivates several
coagulation factors including thrombin, factor IXa and factor Xa. In
addition, thrombin also initiates a negative feedback loop that
possibly limits continued thrombin production by binding
thrombomodulin on endothelial cells and activating Protein C.
Activated Protein C inactivates factor Va and factor VIIIa which are
required for continued thrombin generation. The activation of the PC
anticoagulant pathway is initiated after thrombin generation begins,
suggesting that it might provide a negative feedback mechanism to
limit growth of a developing thrombus. Additionally, activated
platelets release endothieliel cell selective adhesion molecule
(ESAM) which interferes with platelet to platelet adhesion.
Presumably the release of anti thrombotic components at late stages
of platelet activation might stop continued thrombus growth
\cite{ChaDen09,E1,HocJon02,JonMan94a,JonMan94b}.

Prior to the Hockin-Mann's model
\cite{HocJon02}, Mann's group proposed a mathematical model in the form of system of
ordinary differential equations \cite{JonMan94a,JonMan94b} to
describe the TF-initiated pathway reactions based on \emph{in vitro}
experiments. The work in \cite{JonMan94a,JonMan94b} provided a good approximation of
empirical data for blood clot formation. Subsequently, this model
\cite{JonMan94a,JonMan94b} was improved  by including blood
anticoagulants tissue factor pathway inhibitor (TFPI) and
antithrombin III (AT-III), and detailed descriptions of coagulation
enzyme activities \cite{HocJon02}. The new model (termed
Hockin-Mann's model thereafter) accurately predicted the nonlinear
dependence of thrombin generation on the tissue factor and has been
widely used to understand how various coagulation factors interact
with each other and how variations of these factors affect thrombin
production \cite{BPM, BWGMR,E1,HacSer06,KF01,XLMKLCRA,ChaDen09}.

Despite wide applications of Hockin-Mann's model
\cite{HocJon02} and the model described in \cite{JonMan94a,JonMan94b}, structures, characteristics of steady state
solutions of theses models, sensitivity of the models, and uncertainty of
parameter values have
yet to be discussed. %In fact, most coagulation models are rarely
%analyzed.
 So far,  only Lo \emph{et al.} conducted kinetic Monte Carlo
simulations~\cite{LoDen05} using the reaction network described in
this model~\cite{HocJon02} to allow them to accurately simulate
blood coagulation with low concentrations of blood zymogens and
enzymes, which deterministic models often fail to predict.
 Specifically, rates of thrombin production were studies using low TF
concentrations. Simulations revealed that ~0.2pM TF was the critical
concentration to cause 50\% of reactions containing 3-fold diluted
whole blood to generate 0.05 U/ml thrombin by 1 hour \cite{LoDen05}.
Additionally, Luan \emph{et al.} \cite{LZV}  applied sensitivity analysis to their model of  coagulation reactions to identify fragile sites. Using parameter
values from the literature, reactions were identified where small
changes in parameter values would have dramatic effects on thrombin
generation and platelet activation. This analysis \cite{LZV}
 identified reactions involving an interaction of factors IX and VIII
as the most sensitive reactions to small fluctuations of relevant
parameters.

 

%In addition to the other models
%\cite{KhaSem89,JesBel93,JonMan94a,ZarPok96,JonMan94b, HocJon02,
%ZarAta01,BunGen03,KF01,FogTan05,XLMKLCRA,LZV}  have been proposed in
%recent years to integrate various aspects of coagulation reactions.
